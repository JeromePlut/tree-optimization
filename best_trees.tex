\documentclass{article}
\usepackage[T1]{fontenc}
\usepackage[utf8]{inputenc}
\usepackage[margin=20mm]{geometry}
\usepackage{math}
\usepackage{unicode}
\usepackage{booktabs}
\usepackage{newunicodechar}
\usepackage{bbm}
\usepackage{tikz}
\usepackage{graphicx}
\usepackage{subcaption}
\usetikzlibrary{calc}
\newunicodechar{𝟙}{\mathbbm{1}}
\makeatletter
\let\listORI\list
\def\FB@listsettings{%
      \setlength{\itemsep}{0.4ex plus 0.2ex minus 0.2ex}%
      \setlength{\parsep}{0.4ex plus 0.2ex minus 0.2ex}%
      \setlength{\topsep}{0.8ex plus 0.4ex minus 0.4ex}%
      \setlength{\partopsep}{0.4ex plus 0.2ex minus 0.2ex}%
      \@tempdima=\parskip
      \addtolength{\topsep}{-\@tempdima}%
      \addtolength{\partopsep}{\@tempdima}}%
\def\list#1#2{\listORI{#1}{\FB@listsettings #2}}%
\makeatother
      
\def\pcost{ν_\mathrm{p}}
\def\scost{ν_\mathrm{c}}
\def\treefrom#1{\left\{#1\right\}}
\newcounter{proofcase}
\newenvironment{proofcases}{\setcounter{proofcase}{0}%
	\def\item{\stepcounter{proofcase}\medskip
	(\roman{proofcase})~}}{}
\def\gcost{ν}
\definecolor{bleu1}{RGB}{185,217,235}
\definecolor{bleu2}{RGB}{162,178,200}
\definecolor{bleu3}{RGB}{1,66,106}
\definecolor{bleu4}{RGB}{0,43,73}
\definecolor{rouge1}{RGB}{245,218,223}
\definecolor{rouge2}{RGB}{220,134,153}
\definecolor{rouge3}{RGB}{224,62,82}
\definecolor{rouge4}{RGB}{218,41,28}
	

\DeclareMathOperator\Exp{Exp} % Polya exponential
\begin{document}

\section{Optimal trees}


\begin{df}
We recursively define a \emph{weighted tree} (with coefficients in some
group~$G$) as either:
\begin{enumerate}
\item a \emph{leaf}, \emph{i.e.} a single weight~$w ∈ G$; or
\item an \emph{inner node}, i.e. a set of
$k$~\emph{branches}~$\treefrom{b_1, …, b_k}$, for an integer~$k ≥ 2$,
which themselves are weighted trees.
\end{enumerate}
\end{df}

We consider \emph{non-planar} trees,
which means that the set of branches is unordered.
\emph{Trees} are weighted trees such that all leaves have weight~$1$.
The \emph{total weight}~$W(t)$ of a weighted tree is
the sum of the weight of all its leaves;
the weight of a tree is thus simply the count of its leaves.


\subsection{Cost functions}

\begin{df}
We define the \emph{parent cost}~$\pcost(t)$ 
and \emph{sibling cost}~$\scost(t)$
of a weighted tree~$t$ in the following way:
\begin{enumerate}
\item if $t$~is a leaf with weight~$w$,
then $\pcost(t) = \scost(t) = w$;
\item if $t$~is an inner node~$\treefrom{b_1, …, b_k}$
with total weight~$n$,
then
\begin{align}
\pcost(t) = n + ∑ \pcost(b_i); \qquad
\scost(t) = (k-1) n + ∑ \scost(b_i).
\end{align}
\end{enumerate}
\end{df}
In particular, if $t$~is binary then, since $k = 2$ for all inner nodes,
$\scost(t) = \pcost(t)$.

While our main interest lies in simple (non-weighted) trees,
weighted trees can be used as shortcuts for studying them.
Namely, let~$t'$ be the (weighted) tree obtained from a (weighted) tree~$t$
by collapsing an inner branch~$b$ to a single leaf
with weight~$\pcost(b)$;
it is then obvious from the definition that~$\pcost(t') = \pcost(t)$.
The same result also holds for~$\scost$.

This allows replacing a tree~$t$ with a possibly large number of leaves
by a simpler collapsed weighted tree~$t'$ with a small number of leaves,
where the weight of each leaf of~$t'$ represents the cost of
a large sub-tree of~$t$.

\subsection{Optimality}

For any tree~$t$,
the \emph{cost function} associated to~$t$
is the affine map~$f_t: λ ↦ λ \pcost(t) + \scost(t)$.
For any set~$T$ of trees,
we define~$f_T(λ) = \inf \acco{f_t(λ),\; t ∈ T}$;
as an infimum of affine functions, $f_T$ is a concave function.
We say that, for a given~$λ$, a tree~$t ∈ T$ is \emph{optimal}
if its cost is the lowest in~$T$, that is, if $f_t(λ) = f_{T}(λ)$.

Define the points $\gcost(t) = (\pcost(t), \scost(t)) ∈ ℝ^2$,
and the sets $\gcost(T) = \acco{\gcost(t),\; t ∈ T}$
and~$\gcost(T)^+$ as the Minkowski sum of~$\gcost(t)$
and the upper-right quadrant of~$ℝ^2$.
Then $f_T(λ)$ is the dual of the convex hull~$H$ of~$\gcost(T)^+$.
This is also the lower-left part of the convex hull of~$\gcost(T)$)
and therefore all its slopes~$λ_i$ are negative.
More precisely, if a vertex~$\gcost(t_i)$ of~$H$
lies between the segments with slopes~$-λ_{i}$ and~$-λ_{i+1}$,
then for any~$λ ∈ [λ_i, λ_{i+1}]$ and any~$t' ∈ T$
one has~$f_{t'}(λ) ≥ f_{t_i}(λ)$;
\emph{i.e.}, $t_i$ is optimal for any~$λ ∈ [λ_i, λ_{i+1}]$.

\begin{center}\begin{tikzpicture}[scale=.5]
\coordinate(O) at (0,0);
\coordinate(A) at (-2,2);
\coordinate(B) at (4,-1.5);
\coordinate (A1) at (-2,2.5);
\coordinate(T) at (3,1.5);
\draw (A) -- (O) -- (B);
\fill[color=bleu1!30] (O)--(B)-- ++(1,0) |- (A1) -- (A) -- cycle;
\node at (-1,1) {\raise 8pt\rlap{$-λ_{i+1}$}};
\node at (1.5,-.5) {\raise 12pt\rlap{\hskip 1em $-λ_i$}};
\node[color=bleu3] at (B){~\hskip 2em \raise 14pt\rlap{$\gcost(T)^+$}};
\fill (O) circle(3pt);
\node at (O) {\raise 7pt\rlap{~$\gcost(t_i)$}};
\fill (T) circle(3pt);
\node at (T) {\raise 7pt\rlap{~$\gcost(t')$}};
\draw[dashed,rouge3] (-3,1.5) -- (3,-1.5);
\node[color=rouge3] at (-2,1){\lower 9pt\llap{$-λ$~}};
\end{tikzpicture}\end{center}
% Namely, each segment of slope~$λ_{i}$
% between two vertices~$\gcost(t_i)$ and
% Namely, to each vertex~$\gcost(t_i)$ of~$H$,
% located between two segments of slopes~$λ_i ≤λ_{i+1}$,
% corresponds the fact that $t_i$ is optimal for~$λ ∈ [λ_i, λ_{i+1}]$.
% % $f_T(λ) = f_{t_i}(λ)$ for any~$λ ∈ [λ_i, λ_{i+1}]$.

This allows, given a set of~$N$ trees,
to compute the function~$f_T$ in time~$O(N \log(N))$,
using for example Graham's algorithm for the convex hull~\cite{Graham}.

\subsection{Trees with two or three branches}

We now describe the configuration space for
the optimal trees with two or three branches
depending on the parameter~$λ$ above as well as
on the relative sizes of the branches.
For this we start by studying the weighted trees with three leaves
and (without loss of generality) total weight one.
There are only two such trees:
\begin{align}
T_2(n_1,n_2,n_3) = \treefrom{n_1,\treefrom{n_2,n_3}}\qquad\text{and}\qquad
T_3(n_1,n_2,n_3) = \treefrom{n_1,n_2,n_3},
\end{align}
where~$n_1 + n_2 + n_3 = 1$.
We find $\pcost(T_2) = \scost(T_2) = n_1 + 2(n_2 + n_3)$,
$\pcost(T_3) = n_1 + n_2 + n_3$ and~$\scost(T_3) = 2 (n_1 + n_2 + n_3)$.
Then, for any~$τ$, we have
\begin{align}
f_{T_3}(τ) - f_{T_2}(τ)
&= n_1 - τ\pa{1-n_1};
\end{align}
in other words, $f_{T_3}(τ) ≤ f_{T_2}(τ)$
if, and only if, $τ ≥ \frac{n_1}{1-n_1}$.

On the other hand, both trees~$T_3$ and~$T_2$ have symmetries.
Namely, using the symmetry of the tree~$T_3$,
one may assume that $n_1$~is the largest of the three weights,
and in particular that~$n_1 ≥ \frac 13$.
In the same way, for any tree with two branches and at least four leaves,
we can select which branch is collapsed to the $n_1$ leaf
and which branch to the~$\treefrom{n_1, n_2}$ leaves.
Since the cost of such a tree increases with~$n_1$,
the optimal tree is reached when the $n_1$ branch is the lightest branch
from the root of the tree, which implies that~$n_1 ≤ \frac 12$.
Putting all these results together,
we can represent the set of optimal weighted trees with three leaves
as in Figure~\ref{fig:trees3}.

In particular, we deduce the following:
\begin{enumerate}
\item For~$τ ≤ \frac 12$,
the optimal trees are binary;
\item for~$τ ≥ 1$ and~$n ≥ 3$,
the optimal trees with $n$~leaves are at least ternary.
\end{enumerate}
For~$τ ∈ [\frac 12, 1]$, the optimal tree configurations
actually have a mixture of binary and ternary nodes; cf. below.

\begin{figure}
\begin{subfigure}{.5\textwidth}
\centering\hbox{%
\rlap{\includegraphics[width=70mm]{trees3}}%
\smash{\raise 46mm\rlap{$τ$}}}%
\caption{Trees with three leaves}\label{fig:trees3}
\end{subfigure}
\begin{subfigure}{.5\textwidth}
\centering\hbox{%
\rlap{\includegraphics[width=70mm]{trees4}}%
\smash{\raise 46mm\rlap{$τ$}}}%
\caption{Trees with four leaves}\label{fig:trees4}
\end{subfigure}
\caption{Optimal tree configurations with a small number of leaves.
The shading indicates the number of leaves of the
resulting tree.
The crossed-out areas mark the configurations which can be eliminated
by symmetries.
}
\end{figure}

\subsection{Trees with four leaves}
Of the five trees with four leaves, the only one which cannot
be decomposed using the families~$T_2$ and~$T_3$ previously defined
is the flat tree~$T_4(n_1, n_2, n_3, n_4) = \treefrom{n_1, n_2, n_3, n_4}$.

We can compare this tree with
$T'_3(n_1, n_2, n_3, n_4) = \treefrom{n_1, n_2, \treefrom{n_3, n_4}}$
and with~$T'_2(n_1, n_2, n_3, n_4) = \treefrom{\treefrom{n_1, n_2},
\treefrom{n_3, n_4}}$; using the same methods as previously, we find
\begin{enumerate}
\item $f_{T_4}(τ) ≤ f_{T'_3}(τ)$ for~$τ ≥ \frac{n_1 + n_2}{n_3 + n_4}$;
\item $f_{T_4}(τ) ≤ f_{T'_2}(τ)$ for~$τ ≥ 1$.
\end{enumerate}
Moreover, by using the symmetries for~$T_4$ or~$T'_2$,
one may always ensure that~$n_3 + n_4 ≥ \frac 12$ in both cases.
Let~$t$ be a large enough tree with a ternary root,
one may also select to collapse~$t$ into a tree~$T'_3$
by completely collapsing the two lightest branches of~$t$,
thus ensuring that~$n_3 + n_4 ≥ \frac 13$.
We thus obtain the configuration space show
in Figure~\ref{fig:trees4} (where the $x$-coordinate is $n_3+n_4$)
and the following conclusions:
\begin{enumerate}
\item For~$τ ≤ 1$,
the optimal trees with $n$~leaves are (at most) ternary.
\item for~$τ ≥ 2$ and~$n ≥ 4$,
the optimal trees with $n$~leaves are (at least) quaternary.
\end{enumerate}


% \subsection{Optimal trees}
% 
% % For any tree~$t ∈ \ro T(n)$, we define
% % \begin{enumerate}
% % \item the \emph{parent cost} of~$t$:
% % $\pcost(L) = 0$, $\pcost(\treefrom{t_1, …, t_k}) = ∑ \pcost(t_i) + n$.
% % \item the \emph{sibling cost} of~$t$:
% % $\scost(L) = 0$, $\scost(\treefrom{t_1, …, t_k}) = ∑ \scost(t_i) + (k-1) n$.
% % \item the \emph{geometric cost} of~$t$
% % as the point~$\gcost(t) = (\pcost(t), \scost(t)) ∈ ℝ^2$.
% % \end{enumerate}
% % In particular, if $t$~is a binary tree
% % then $\scost(L) = \pcost(L)$.
% % 
% For any tree~$t$,
% the \emph{cost function} associated to~$t$
% is the affine map~$f_t: λ ↦ λ \pcost(t) + \scost(t)$.
% For any set~$T$ of trees,
% we define~$f_T(λ) = \inf \acco{f_t(λ),\; t ∈ T}$;
% as an infimum of affine functions, $f_T$ is a concave function.
% In fact, it is the dual of the convex hull
% of the set~$\acco{ \gcost(t),\; t ∈ T}$:
% 
% \begin{prop}\label{prop:convex}
% Let~$(t_i)$ be the set of points on the lower-left convex hull of
% $G = \acco{\gcost(t),\: t ∈ T}$,
% \emph{i.e.} those points~$t$ such that $G$ intersects non-trivially
% the upper-right quadrant originated at~$t$.
% Assume that $(t_i) = (t_1, …, t_n)$ is finite
% and that they are sorted by decreasing value of~$\pcost(t_i)$;
% define $λ_i = \frac{\scost(t_{i+1})-\scost(t_i)}{\pcost(t_{i+1})-\pcost(t_i)}$.
% Then, for any~$τ ∈ [λ_{i-1}, λ_i]$, $f_T(τ) = f_{t_i}(τ)$;
% that is, the tree~$t_i$ minimizes $f_{t}(τ)$ for all~$t ∈ T$.
% \end{prop}
% 

\subsection{Counting trees}
The single-child condition ensures that $\ro T(n)$~is a finite set for any~$n$.
Let~$T$ be the generating series $T(z) = ∑ \abs{\ro T(n)}\: z^n$;
then $T$~satisfies the functional equation
\begin{align}
T(z) = x + \Exp(T)(z) - T(z) - 1,
\end{align}
where $\Exp(T)$ is the Polya exponential, defined by
$\Exp(f)(z) = ∑_{n≥0} f(x^n)/n$.

(TODO)


\subsection{Cost functions for some families of trees}

In this section we study a few given families of trees,
parametrized by their number of leaves~$n$;
empirically, these are the optimal families for~$λ$ small.
While it is possible to check this for small~$n$,
the exponential growth of the number of trees
(see previous section)
prevents this exploration from going beyond, say, $n = 25$.
We believe it reasonable to conjecture that
these families remain optimal in the given domains
among all trees for all values of~$n$.


\subsubsection{Tree-like integration and derivation}

We now introduce a tool which will enable us to
systematically compute the parent and sibling costs
for some infinite families of trees.
The main result is Prop.~\ref{prop:interpolation},
which allows the computation of tree-like integrals
of finite sequences of integers.


Let~$d ≥ 2$ be an integer.
For any function~$f: ℕ → ℝ$ we define the
\emph{tree-like derivative} of~$f$ by
\begin{align}\label{eq:def-D}
\go D(f)(n) &= f(n) - ∑_{i=0:d-1} f\pa{\floor{\frac{n+i}{d}}}.
\end{align}
One easily checks that $\go D(f)(0) =\go D(f)(1) = (1-d) f(0)$.
We also define the \emph{tree-like integral}
of any function~$f$ such that~$f(1) = f(0)$ by
\begin{align}\label{eq:def-I}
\go I(f)(n) &= \begin{cases}
\frac{1}{1-d}f(0)  & \text{for~$n = 0$ or~$n = 1$},\\
\displaystyle f(n) + ∑_{i=0:d-1} \go I_d(f)\pa{\floor{\frac{n+i}{d}}}
	& \text{for $n ≥ 2$.}
\end{cases}
\end{align}
% (The condition~$f(1) = f(0)$ is necessary for
% the definition of~$\go I(f)(1)$ to make sense).
When the degree~$d$ is not obvious we shall write it as a subscript,
thus $\go D_d$, $\go I_d$.


The operator~$\go I$ is a right inverse of~$\go D$;
that is, $\go D ∘ \go I$ is the identity
on functions such that~$f(0) = f(1)$.
In fact, $\go D$ and~$\go I$ are inverse linear automorphisms of this space.

% We also define the following sequences:
% $𝟙(n) = 1$; $J(n) = 1$ for~$n = 0$ and~$n$ for~$n ≥ 1$;
% $U(n) = 1$ for~$n ≤ 1$ and~$0$ for~$n ≥ 2$;
% $E_d(n) = e ⋅ n - \frac{d^e}{d-1}$ for~$n ∈ [d^e, d^{e+1}]$
% and~$E_d(0) = E_d(1) = \frac{1}{1-d}$.
% Then one easily checks the following relations:
% \begin{align}
% \go D(𝟙) = \go I(𝟙) = (1-d)⋅𝟙; \qquad
% % \go D(J) = -U; \qquad
% \go I(J) = E_d.
% \end{align}


\medbreak

We now introduce the notation for affine interpolation
as it will be used in Prop.~\ref{prop:interpolation}.
For~$x ∈ ℝ$, let~$Λ(x) = \frac 12 \abs{x-1} + \frac 12 \abs{x+1} -
\abs{x}$; then $Λ$ is continuous, piecewise affine,
and~$Λ(0) = 1$ while~$Λ(n) = 0$ for~$n ∈ ℤ⧵ \acco{0}$.
For any sequence~$(y_i)$ of integers,
we define the function~$\widehat{y}(x) = ∑ y_i Λ(x-i)$;
this function is also continuous, piecewise affine,
and interpolates~$(y_i)$.

\begin{prop}\label{prop:interpolation}
Let~$(y_i)$ be a sequence of real numbers such that~$y_0 = y_1$
and let~$(z_i) = \go D(y)$;
assume there exists an integer~$c$ such that~$z_i = 0$ for~$i ≥ d^c$.
Let~$\widehat{y}$ be the piecewise affine function
interpolating the values~$(y_{d^{c-1}}, …, y_{d^{c}-1},
y_{d^c} = d y_{d^{c-1}})$.
Then, for any integers~$e$ and~$n ∈ [d^{c+e}\!,\: d^{c+e+1}]$:
\begin{align}
y_n &= n^e\, \widehat{y}\pa{n/d^e}.
\end{align}
% Let~$(y_i)$ be a sequence of real numbers
% such that~$y_0 = y_1$ and~$y_i = 0$ for~$i ≥ d^c$.
% Let~$f$ be the unique function interpolating~$(y_i)$
% which is affine on each interval~$[i, i+1]$,
% with the additional condition that~$f(d^c) = d f(d^{c-1})$.
% Then, for any integer~$n ∈ [d^{c+e}, d^{c+e+1}]$,
% $\go I(y)(n) = d^e f(n/d^e)$.
\end{prop}

\begin{proof}
We first note that~$z_{d^{c}} = 0$ implies~$y_{d^c} = d y_{d^{c-1}}$.
We prove the result by induction on~$e$; the case~$e = 0$ is obvious.
Let~$n ∈ [d^{c+e+1}, d^{c+e+2}]$
and write~$n = a d + b$ be the Euclidan division of~$n$ by~$d$;
then the relation~$z_{n} = 0$ implies
\begin{align}
y_n &= ∑ y_{\floor{\frac{n+i}{d}}}\\
 &= (d-b) y_{a} + b y_{a+1}\\
 &= d^e \pa{(d-b) \widehat{y}(a/d^e) \,+\, b \widehat{y}((a+1)/d^e)}.
% \go I(y)(n) &= y(n) + ∑ \go I(y)\pa{\floor{\frac{y+i}{d}}}\\
% &= 0 + (d-b) \go I(y)(a) + b \go I(y)(a+1)\\
% &= d^e \pa{(d-b) f\pa{\frac{a}{d^e}} + b f\pa{\frac{a+1}{d^e}}}.
\end{align}
Since $\widehat{y}$~is affine on
the interval~$\cro{\frac{a}{d^e}, \frac{a+1}{d^e}}$,
the last value is also equal to $\widehat{y}\pa{a+b/d}{d^e}$.
\end{proof}


\begin{prop}
Define the functions~$J(n) = \max(n, 1)$
and, for any~$n ∈ [d^e, d^{e+1}]$, $E_d(n) = (e+1)⋅n - \frac{d^{e+1}}{d-1}$.
Then
\begin{align}
\go I_d(J) = E_d.
\end{align}
\end{prop}
% $E_d(n) = e ⋅ n - \frac{d^e}{d-1}$ for~$n ∈ [d^e, d^{e+1}]$
% and~$E_d(0) = E_d(1) = \frac{1}{1-d}$.
% We check that $\go I_d(J) = E_d$.

% \begin{prop}\label{prop:IU}
% Define~$f_d(x) = \frac{d-2}{d-1}-x$ for~$x ≤ 2$
% and~$f_d(x) = \frac{-d}{d-1}$ for~$x ≥ 2$.
% Then, for any integer~$n ≥ 1$, if~$e = \floor{\log_d(n)}$,
% $\go I_d(J)(n) = d^e f_d(n/d^e)$.
% \end{prop}

\subsubsection{Regular families of trees}

Let~$(d_n)$ be a sequence of integers such that~$d_0 = d_1 = 0$.
The \emph{regular family of trees} with degrees~$(d_n)$
is the family of trees~$(T(n))$ with $n$ leaves
such that, for~$n ≥ 2$,
$T(n) = \acco{T\pa{\floor{\frac{n+i}{d(n)}}},\, i=0…d(n)-1}$.
We can compute the parent costs and sibling costs
of some regular families using Prop.~\ref{prop:interpolation}.

\begin{prop}
Let~$d$ be an integer and define the family of trees~$T_d$
as the regular family of trees with degrees~$\min(n,d)$ for~$n ≥ 2$.
Then the parent cost and sibling cost of~$T_d(n)$ are
\begin{align}
\pcost(T_d(n)) &= E_d(n) - f_d(n) = \begin{cases}
(e+2) n - 2 d^e & \text{for~$n ∈ [d^e, 2 d^e]$,}\\
(e+1) n & \text{for~$n ∈ [2 d^e, d^{e+1}]$.}
\end{cases}\\
\scost(T_d(n)) &= ((d-1)e + 2b)⋅n - b(b+1)⋅d^e
\quad \text{for $n ∈ [b⋅d^e, (b+1)⋅d^e]$, $b ∈ \acco{1, …, d-1}$.}
\end{align}
\end{prop}


\begin{proof}
Let~$u = \pcost(T_d) - E_d$.
Then we check that $\go D_d(u)(n) = 0$ for any~$n ≥ d$.
Thus, by Prop.~\ref{prop:interpolation}, for any~$n ∈ [d^e, d^{e+1}]$,
$u(n) = d^e \widehat{u}(n/d^e)$
where~$\widehat{u}$ is the affine interpolation of~$u$ on~$\acco{1,…,d}$.
then, by Prop.~\ref{prop:interpolation},
It now suffices to check that $\widehat{u}(x) = \frac{d-2}{d-1} - x$
for~$x ∈ [1,2]$ and~$-\frac{d}{d-1}$ for~$x ∈ [2,d]$.

The computation of~$\scost(T_d)$ uses the same reasoning.
Let~$v = \scost(T_d) - (d-1) E_d$; then $\go D_d(v) = 0$ for~$n ≥ d$.
For~$n ∈ \acco{1,…,d}$, we find $v_n = n(n-1) - (d-1)E_d(n) = n^2 -d n + d$,
so that, for~$x ∈ [b,b+1]$, $\widehat{v}(x) = (2b-d+1) x + (d-b^2-b)$.
\end{proof}

Using the same process (which can easily be automated),
we can compute the parent and sibling costs
for some explicitly defined regular families of trees.

\begin{prop}
Let~$T_{2,3}$ be the regular family of trees
with degrees~$d_2 = 2$, $d_3 = 3$ and~$d_n = 2$ for~$n ≥ 4$. Then:
\begin{align}
\pcost(T_{2,3}(n)) &= \begin{cases}
e⋅ n & \text{for $n ∈ [2^e, \frac{3}{2} ⋅ 2^e]$,}\\
(e+4)⋅n - 6 ⋅ 2^e & \text{for $n ∈ [\frac{3}{2}⋅2^e, 2^{e+1}]$.}\end{cases}\\
\scost(T_{2,3}(n)) &= \begin{cases}
(e+3)⋅n - 3⋅2^e & \text{for~$n ∈ [2^e, \frac{3}{2}⋅2^e]$,}\\
(e+1)⋅n & \text{for~$n ∈ [\frac{3}{2}⋅2^e, 2^{e+1}]$.}\end{cases}
\end{align}
\end{prop}

\begin{prop}
Define the following regular families of trees:
\begin{enumerate}
\item $T_{3,2}$ with degrees~$d_2 = 2$, $d_4 = 2$ and~$d_n = 3$ otherwise;
\item $T_3$ with degrees $d_2 = 2$ and $d_n = 3$ otherwise;
\item $T_{3,4}$ with degrees $d_2 = 2$, $d_4 = 4$ and~$d_n = 3$ otherwise.
\end{enumerate}
The following table summarizes the costs of these families:
\begin{align}
\begin{array}{lllll}
\toprule
\text{for $n$ in:} &
{}[3^e, 4⋅3^{e-1}] &
{}[4⋅3^{e-1}, 5⋅3^{e-1}] &
{}[5⋅3^{e-1}, 2⋅3^e] &
{}[2⋅3^e, 3^{e+1}]\\
\midrule
\pcost(T_{3,2}(n)): & (e+4)⋅n-4⋅3^e & e⋅n-4⋅3^{e-1} &
 (e+2)⋅n - 2⋅3^e & (e+1)⋅n\\
\scost(T_{3,2}(n)): & 2e⋅n & (2e+4)⋅n - 16⋅3^{e-1} &
 (2e+2)⋅n - 2⋅3^e & (2e+4)⋅n - 6⋅3^e \\
\midrule
\pcost(T_{3}(n)): & & (e+2)⋅n-2⋅3^e & & (e+1)⋅n \\
\scost(T_{3}(n)): & & (2e+2)⋅n-2⋅3^e & & (2e+4)⋅n-6⋅3^e\\
\midrule
\pcost(T_{3,4}(n)): & e⋅n & (e+4)⋅n-16⋅3^{e-1} &
	(e+2)⋅n-2⋅3^e & (e+1)⋅n \\
\scost(T_{3,4}(n)): & (2e+4)⋅n - 4⋅3^e & 2e⋅n + 4⋅3^{e-1} &
	(2e+2)⋅n-2⋅3^e & (2e+4)⋅n - 6⋅3^e \\
\bottomrule
\end{array}
\end{align}
\end{prop}


\subsubsection{Averages}

Let~$μ$ be the geometric distribution on~$⟦1,∞⟦$
with parameter~$ρ ∈ [0,1[$: $μ(n) = ρ^{n-1}(1-ρ)$ for any~$n ≥ 1$.

We can compute the expected value, according to~$μ$,
for the cost of the regular tree~$T_d$:


\begin{proof}

\begin{align}
𝔼(\pcost(T_d))
&= ∑_{e≥0} \pa{∑_{n=d^e}^{2 d^e-1} μ(n) \pa{(e+2) n - 2 d^e}
	+ ∑_{n=2 d^e}^{d^{e+1}-1} μ(n)\pa{(e+1) n} } \\
&= ∑_{e≥0} \pa{ (e+1) ∑_{n=d^e}^{d^{e+1}-1} n μ(n)
	+ ∑_{n=d^e}^{2 d^e-1} μ(n) \pa{n - 2 d^e}}\\
&= ∑_{e≥0} \pa{ (e+2) ∑_{n=d^e}^{2 d^e-1} n μ(n)
	+ (e+1) ∑_{n=2 d^e}^{d^{e+1}-1} n μ(n)
	- 2 d^e ∑_{n=d^e}^{2 d^e-1} μ(n)}\\
&= (1-ρ) ∑_{e≥0} \pa{ (e+2) ∑_{n=d^e}^{2 d^e-1} n ρ^n
	+ (e+1) ∑_{n=2 d^e}^{d^{e+1}-1} n ρ^n
	- 2 d^e ∑_{n=d^e}^{2 d^e-1} ρ^n}\\
% &= (1-ρ) ∑_{e≥0} \big( (e+1) \frac{1}{(ρ-1)^2}
% 	\pa{(d^{e+1}-1) ρ^{d^{e+1}} - d^{e+1} ρ^{d^{e+1}-1}
% 		- (d^{e}-1) ρ^{d^e+1} + d^e ρ^{d^e}}\\
% &
% 	- 2 d^{2e} + \frac{1}{(ρ-1)^2} \pa{
% 	(2 d^e-1) ρ^{2 d^e} - 2 d^e ρ^{2 d^e-1}
% 	- (d^e-1) ρ^{d^e+1} + d^e ρ^{d^e}}\big)\\	
\end{align}
By deriving the geometric sum $∑ X^n$ we find, for~$X ≠ 1$:
\begin{align}
∑_{n=a}^{b-1} n X^n
&= \frac{1}{(X-1)^2}
	\pa{(b-1) X^{b+1} - b X^{b} - (a-1) X^{a+1} + aX^a}.
\end{align}
Using the shorthands $a = d^e$, $b = 2 d^e$ and~$c = d^{e+1}$
we find:
\begin{align}
(1-ρ)^2 ∑_{n=a}^{b-1} n ρ^n
	&= (b-1) ρ^{b+1} - b ρ^{b} - (a-1) ρ^{a+1} + a ρ^a;\\
(1-ρ)^2 ∑_{n=b}^{c-1} n ρ^n
	&= (c-1) ρ^{c+1} - c ρ^{c} - (b-1) ρ^{b+1} + b ρ^b;\\
(1-ρ)^2 ∑_{n=a}^{b-1} \hphantom{n}ρ^n &= (1-ρ) (ρ^a - ρ^b)
	\quad =\quad ρ^{b+1} - ρ^{b} - ρ^{a+1} + ρ^a;
\end{align}
and hence, grouping all terms together by powers of~$ρ$:
\begin{align}
(1-ρ) 𝔼(\pcost(T_d))
&= ∑_{e ≥ 0} \pa{
	(e+1)(c-1) ρ^{c+1} - (e+1)c ρ^c - ρ^b - e(a-1) ρ^{a+1}+  2 ρ^{a+1} + ea ρ^a
	}
\end{align}
When summing over all~$e$, the terms in~$ρ^a$ cancel those with~$ρ^c$
and likewise the terms~$-e(a-1) ρ^{a+1}$ cancel those with~$ρ^{c+1}$,
and it remains
\begin{align}
(1-ρ) 𝔼(\pcost(T_d)) &= ∑_{e ≥ 0} \pa{2 ρ^{d^e+1} - ρ^{2 d^e}}.
\end{align}

On the other hand:
\begin{align*}
𝔼(\scost(T_d)) &=
(1-ρ)∑_{e ≥ 0} ∑_{b=1}^{d-1} ∑_{n=b d^e}^{(b+1) d^e-1}
	ρ^n \pa{((d-1)e + 2b) n - b(b+1) d^e}\\
&=(1-ρ)∑_{e≥0} ∑_{b=1}^{d-1} ((d-1)e + 2b) ∑ n ρ^n - b(b+1) d^e ∑ ρ^n\\
\end{align*}
\end{proof}


\end{document}
