\documentclass{article}
\usepackage[T1]{fontenc}
\usepackage[utf8]{inputenc}
\usepackage[margin=20mm]{geometry}
\usepackage{math}
\usepackage{unicode}

\def\pcost{ν_\mathrm{p}}


\DeclareMathOperator\Exp{Exp} % Polya exponential
\begin{document}

\section{Optimal trees}

\begin{df}
Let~$\ro T(n)$ be the set of all non-planar trees
with $n$ leaves such that no node has a single child.
\end{df}

\subsection{Counting trees}
The single-child condition ensures that $\ro T(n)$~is a finite set.
Let~$T$ be the generating series
$T(z) = ∑ \abs{\ro T(n)}\: z^n$;
then $T$~satisfies the functional equation
\begin{split}
T(z) = x + \Exp(T)(z) - T(z) - 1,
\end{split}
where $\Exp(T)$ is the Polya exponential, defined by
$\Exp(f)(z) = ∑_{n≥0} f(x^n)/n$.

\subsection{Optimal trees}

We use the following syntax for trees: $L$ is the tree with only one
leaf, and given trees $t_1, …, t_n$, we write $\acco{t_1, …, t_n}$
for the tree where $(t_i)$ are the branches of the root node.
Note that, since our trees are non-planar, this tree
does not depend on the ordering of the trees~$t_i$.

We define the following tree families:
\begin{enumerate}
\item $T_2(1) = L$ and $T_2(n) = \acco{T_2(\floor{n/2}),
T_2(\ceil{n/2})}$ for~$n ≥ 2$;
\item $T_{23}(1) = L$,
$T_{23}(3⋅2^k) = \acco{T_{23}(2^k), T_{23}(2^k), T_{23}(2^k)}$,
and $T_{23}(n) = \acco{T_{23}(\floor{n/2}), T_{23}(\ceil{n/2})}$
otherwise.
\end{enumerate}
In particular, one easily checks that $T_{23}(n) = T_2(n)$
whenever $n$~is a power of two.


\begin{prop}
Let~$n$ be an integer and~$b = \floor{\log_2(n)}$.
Then:
\begin{enumerate}
\item $\pcost(T_{23}(n)) =
\begin{cases}
b⋅i \,+\, 4(i - 3⋅2^{b-1}) & \text{if $3 ⋅ 2^{b-1}< n < 2^{b+1}$},\\
b⋅i & \text{if $2^{b} ≤ n ≤ 3⋅2^{b-1}$.}
\end{cases}
\end{enumerate}
\end{prop}


\begin{proposition}

\end{proposition}

\end{document}
